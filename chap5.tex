%% This is an example first chapter.  You should put chapter/appendix that you
%% write into a separate file, and add a line \include{yourfilename} to
%% main.tex, where `yourfilename.tex' is the name of the chapter/appendix file.
%% You can process specific files by typing their names in at the 
%% \files=
%% prompt when you run the file main.tex through LaTeX.

%% Topics:
%% Frontend modifications
%% Backend modifications
%% Implementation overview of default handlers 
%% Implementation of DSL and Authn function
%%   How this plays with:
%%   - default input getters
%%   - context retrieval 

\chapter{WebAuthn Firewall Implementation}\label{Chap:WebauthnFirewallImplementation}

The WebAuthn firewall presented in this thesis is implemented in the Go programming language. This chapter describes how some of the critical components of \sys{}'s design described in Chapter~\ref{Chap:WebauthnFirewallDesign} are implemented. The following Table~\ref{Table:ImplementationFootprint} outlines the approximate footprint of each component discussed in this chapter.

\begin{table}[h]
\centering

\begin{tabular}{ m{5cm} m{3cm}  } 
 \hline
 Firewall Component & Lines of Code \\ 
 \hline \hline

 WebAuthn Verification & 126 \\ \hline

 Default Handlers & 309 \\ \hline

 Domain Specific Language & 478 \\ \hline

\end{tabular}
\caption{The domain specific language \lstinline{Get} type operations. These affect the format string if invoked at the top level within \lstinline|Authn|.}
\label{Table:ImplementationFootprint}
\end{table}

%% 
%% \iffalse
%% \section{Webauthn JavaScript Library}

%% The webauthn JavaScript library provides easy access to the webauthn setup and verification life-cycle for the frontend. It exposes a few functions which take care of the boilerplate code. 

%% The functions for setup are:

%% \begin{lstlisting}
%% const retrieveWebauthnOptions_FormField = async (form_id, field_name) => { ... }
%% const retrieveWebauthnOptions_Cookie = async (src_cookie) => { ... }
%% const retrieveWebauthnOptions_URL = async (form_id, src_url) => { ... }
%% \end{lstlisting}

%% They all access the setup data from the firewall by their respective payload types. The \lstinline{FormField} reads the data from a form in the current HTML page. The \lstinline{Cookie} reads the data from some cookie. And the most commonly used \lstinline{URL} performs a POST request and reads the setup data from the response.

%% The functions for registration are:

%% \begin{lstlisting}
%% const registrationFinish_URL = async (credentialCreateOptionsFromServer, finish_url, form_id) => { ... }
%% const registrationFinish_PostFn = async (credentialCreateOptionsFromServer, post_fn) => { ... }
%% \end{lstlisting}

%% They send over the hardware authenticator's credentials to the firewall during the registration event after the setup has been performed. Each performs the same processing procedure on the \lstinline{credentialCreateOptionsFromServer} webauthn options before issuing a POST request over to the firewall. The functions take the \lstinline{credentialCreateOptionsFromServer}, manipulate the encoding, get the hardware authenticator's credential and then package it into a clean JavaScript object. Fetching the hardware authenticator's credential is done with the web-browser exposing an API. The web-browser handles the interaction with the hardware device.

%% \begin{lstlisting}
%% const credential = await navigator.credentials.create({
%%   publicKey: publicKeyCredentialCreateOptions
%% });
%% \end{lstlisting}

%% The \lstinline{PostFn} function lets the caller specify a function, \lstinline{post_fn}, which performs the POST request instead of from the library as is the case with the \lstinline(URL) function.

%% The functions for webauthn transaction authentication:

%% \begin{lstlisting}
%% const attestationFinish_URL = async (credentialRequestOptionsFromServer, finish_url, form_id) => { ... }
%% const attestationFinish_PostFn = async (credentialRequestOptionsFromServer, post_fn) => { ... }
%% \end{lstlisting}

%% These functions are implemented nearly identically as their registration counterparts. The \lstinline{PostFn} function relies on a supplied \lstinline{post_fn} to issue the POST request whereas the \lstinline{URL} does it as a part of the library code. The primary difference is that the web-browser's API call is different.

%% \begin{lstlisting}
%% const assertion = await navigator.credentials.get({
%%     publicKey: transformedCredentialRequestOptions,
%% });
%% \end{lstlisting}

%% The \lstinline{transformedCredentialRequestOptions} contains all of the necessary transaction authentication setup information described in Section~\ref{Sec:TransactionAuthenticationSetup}, such as the challenge and the authentication message. This web-browser API call passes on that information to the hardware authenticator device. If the user confirms the operation on the hardware device, an \lstinline{assertion} is returned with the authentication data described in Section~\ref{Sec:WebauthnFirewallVerification}.
%% \fi
%% 

\section{WebAuthn Verification}\label{Sec:WebauthnVerification}

The cryptographic verification of a WebAuthn transaction authentication event uses a slightly modified Go WebAuthn library~\cite{webauthn-library}. This library exposes two functions, \lstinline{BeginLogin} and \lstinline{FinishLogin}, which set up and validate a WebAuthn two-factor authentication event respectively. The \lstinline{FinishLogin} function is modified to support the transaction authentication extension. It checks if the extensions object received from the hardware authenticator exactly match an expected extensions object generated by the firewall.

%% 
%% \iffalse
%% The \lstinline{BeginLogin} call is invoked during the webauthn setup stage and is used straightforwardly.

%% \begin{lstlisting}
%% // Generate the webauthn `options` and `sessionData`
%% options, sessionData, err := webauthnAPI.BeginLogin(wuser)
%% if r.HandleError(w, err) {
%%   return
%% }

%% \end{lstlisting}

%% The \lstinline{wuser} is a webauthn user entry from the firewall's database store. It contains typical information such as the user's ID and public key credential. Returned are the \lstinline{options} that get sent back to the frontend and eventually the hardware authenticator as the result of the setup stage. The \lstinline{sessionData} is remembered by the firewall for the current HTTP request as described in Section~\ref{Sec:TransactionAuthenticationSetup}.

%% The \lstinline{FinishLogin} call is invoked more selectively. If a user does not have webauthn currently enabled, that is they do not have a registered hardware device with the web service, then routes that are webauthn protected should not check for webauthn for that user's requests. The following is a simplification of the webauthn transaction verifying code.

%% \begin{lstlisting}
%% // See if the user has webauthn enabled
%% isEnabled := db.WebauthnStore.IsUserEnabled(db.QueryByUserID(userID))

%% // Perform a webauthn check if webauthn is enabled for this user
%% if isEnabled {
%%   // Parse the form-data to retrieve the `http.Request` information
%%   assertion, err := r.Get_WithErr("assertion")

%%   // Get the `authnText` to verify against
%%   authnText := getAuthnText(r)

%%   // Populate the `extensions` with the `authnText`
%%   extensions := make(protocol.AuthenticationExtensions)
%%   extensions["txAuthSimple"] = authnText

%%   // Load the session data
%%   sessionData, err := sessionStore.GetWebauthnSession("authentication", r.Request)

%%   // Perform the webauthn check
%%   _, err = webauthnAPI.FinishLogin(wuser, sessionData, extensions, assertion)
%%   if err != nil {
%%     return false
%%   }
%% }
%% \end{lstlisting}

%% The code first checks if the current user has webauthn enabled. Then only if so does it proceed to extract the authentication data from \lstinline{r.Get_WithErr("assertion")}, generate the authentication message from \lstinline{getAuthnText} to validate against, retrieve the saved \lstinline{sessionData} and finally perform the cryptographic check with \lstinline{FinishLogin}.
%% \fi
%% 

\section{Default Handlers}

As discussed in~\ref{Sec:DefaultHandlers}, every web application with WebAuthn transaction authentication must support a list of core WebAuthn operations. \sys{} secures them transparently with a collection of default handlers. They can be grouped in Table~\ref{Table:Implementation_DefaultHandlers} by their similar implementations.

%% 
%% \iffalse
%% The webauthn firewall secures a number of core webauthn operations detailed in~\ref{Sec:DefaultHandlers}. Many of them can be grouped in Table~\ref{Table:Implementation_DefaultHandlers} by their similar implementations.
%% \fi
%% 

\begin{table}[h]
\centering

\begin{tabular}{ m{3cm} m{11cm}  } 
 \hline
 Function & Description \\ 
 \hline \hline

 \lstinline|webauthnIsEnabled| & Queries the firewall's database to determine if a \lstinline|username| has WebAuthn enabled or not. \\ \hline

 \lstinline|beginRegister|, \lstinline|beginLogin|, \lstinline|beginAttestation| & Functions that setup their respective operations as described in Section~\ref{Sec:WebauthnVerification}. \\ \hline

 \lstinline|finishRegister|, \lstinline|finishLogin| & Validates WebAuthn public key credentials. If successful, \lstinline|finishRegister| saves the credentials in the firewall's database, whereas \lstinline|finishLogin| allows the login to continue. \\ \hline

 \lstinline|disableWebauthn| & Validates WebAuthn public key credential and authentication message \lstinline|"Confirm disable WebAuthn for {{ username }}"|. If successful, it deletes the credential from the firewall's database. \\ \hline

\end{tabular}
\caption{The default handlers included with the WebAuthn firewall.}
\label{Table:Implementation_DefaultHandlers}
\end{table}

%% 
%% \iffalse
%% \begin{itemize}[nosep]
%% \item \lstinline{webauthnIsEnabled} is the simplest. It retrieves the \lstinline{username} to check from the contents of the HTTP request. It then performs a database query on the user and returns whether they have webauthn enabled or not.

%% \item \lstinline{beginRegister}, \lstinline{beginLogin} and \lstinline{beginAttestation} all are used during setup. The \lstinline{beginRegister} performs a call to the webauthn Go library function \lstinline{webauthnAPI.BeginRegistration}. The \lstinline{beginLogin} and \lstinline{beginAttestation} call \lstinline{webauthnAPI.BeginLogin} and follow the setup described in Section~\ref{Sec:WebauthnVerification}. 

%% Whenever \lstinline{beginLogin} is called, the user has not logged into their account yet and this function should prepare a simple two-factor authentication event. So the \lstinline{r.Get_WithErr("username")} contained in the HTTP request is used to identify the current user and no authentication text is used. The \lstinline{beginAttestation} uses \lstinline{r.GetUserID()} described in Section~\ref{Sec:ConfigurationParameters} to get the current user and also begins a transaction authentication event with the authentication message in \lstinline{r.Get_WithErr("auth_text")}.

%% All of these library calls return session data to be saved locally at the firewall and webauthn options data to be returned in response to the request.

%% \item \lstinline{finishRegister}, \lstinline{finishLogin} and \lstinline{disableWebauthn} all call a finishing function. The \lstinline{finishRegister} calls \lstinline{webauthnAPI.FinishRegistration} which validates the HTTP register request and credentials. If all passes, then the credentials are saved in the firewall for the current user. The \lstinline{finishLogin} and \lstinline{disableWebauthn} both perform the same verification described in Section~\ref{Sec:WebauthnVerification} by calling \lstinline{webauthnAPI.FinishLogin}. The \lstinline{finishLogin} does not check any transaction authentication message since it is for simple two-factor authentication. The \lstinline{disableWebauthn} checks the authentication message \lstinline|"Confirm disable webauthn for {{ username }}"|.

%% \end{itemize}

%% \fi
%% 

%% 
%% \iffalse
%% first check if the user has webauthn enabled and then call \lstinline{webauthnAPI.BeginLogin}. Similarly as in Section~\ref{Sec:WebauthnVerification}, these Go library functions return session data to be saved locally at the firewall and a options data to be returned as a response to this request.
%% \fi
%% 

\section{Domain Specific Language}

\sys{} secures most routes using the domain specific language within the \lstinline{Authn} function described in Section~\ref{Sec:DomainSpecificLanguage}. Any domain specific program has a state and an output container. A \lstinline{scope} hash table holds the state of the program --- all of the local variables which may be set and accessed. The has table maps strings representing variable names to their respective contained values. The output container is a \lstinline{formatVars} array. This array stores the values to apply sequentially to the format tags included in the first argument of \lstinline{Authn}.

%% 
%% \iffalse
%%  local variables which may be set and accessed. They are stored in a \lstinline{scope} hash table which maps strings representing variable names to their respective contained values. As the domain specific program runs, it populates a \lstinline{formatVars} array. This array stores the values to apply sequentially to the format tags included in the first argument of \lstinline{Authn}.
%% \fi
%% 

%% 
%% \iffalse
%% has a \lstinline{scope} hash map and a \lstinline{formatVars} array. The \lstinline{scope} maps strings standing for variable names to their respective contained values. The \lstinline{formatVars} array stores the values to apply sequentially to the format tags included in the first argument of \lstinline{Authn}.
%% \fi
%% 

The first argument of the \lstinline{Authn} function is the format string. The second argument and onward are the top-level DSL operations. Only the \lstinline{Get} type operations listed in Table~\ref{Table:DSL_GetterOperations} which appear as top-level operations may affect the resulting authentication message. All other occurrences, such as arguments to other domain specific operations, simply return their respective values. 

%% 
%% \iffalse
%% Only those operations that appear at the top-level of the \lstinline{Authn} may affect the authentication message. 
%% \fi
%% 

Each DSL operation must implement an \lstinline{execute} and \lstinline{retrieve} function. The \lstinline{Authn} executes the domain specific program line-by-line by calling \lstinline{execute} of the top-level operations in order. Since \lstinline{execute} is only called on top-level operations, which may affect the authentication message, it has access to both the \lstinline{scope} and the output \lstinline{formatVars} array. The function type of \lstinline{execute} is produced in Code Snippet~\ref{code:executeFunctionType}.

%% 
%% \iffalse
%%  lists the function type of \lstinline{execute} is as follows:

%% The \lstinline{execute} function of a DSL operation only gets called whenever that operation is at the top-level. As a result, the operation has access to the \lstinline{formatVars} array. The type is as follows:

%% Since the top-level operations may affect the authentication message, they have access to the \lstinline{formatVars} array. 
%% \fi
%% 

\begin{lstlisting}[float=h,label=code:executeFunctionType,caption=The function type of the \lstinline{execute} function.]
execute(r *ExtendedRequest, 
        scope scopeContainer, 
        formatVars *[]interface{})
\end{lstlisting}

Table~\ref{Table:DSL_GetterOperations} outlines the \lstinline{Get} type operations. When \lstinline{execute} is called on one such operation, it typically appends to the \lstinline{formatVars} array. The values of \lstinline{formatVars} substitute the format tags in the format string in order. Table~\ref{Table:DSL_SetterOperations} outlines the \lstinline{Set} type operations and calling \lstinline{execute} typically adds or sets a new variable to the \lstinline{scope} hash table.

%% 
%% \iffalse
%% A call to \lstinline{execute} can use the request being authenticated and current \lstinline{scope} to add or remove variables from the \lstinline{scope} or append to the \lstinline{formatVars} array. Table~\ref{Table:DSL_GetterOperations} outlines the \lstinline{Get} type operations that typically modify the \lstinline{formatVars} array. Table~\ref{Table:DSL_SetterOperations} outlines the \lstinline{Set} type operations that typically add new variables to the \lstinline{scope}.
%% \fi
%% 

%% 
%% \iffalse
%% Generally, the \lstinline{Get} type operations just modify the \lstinline{formatVars} array and the \lstinline{Set} type operations just modify the \lstinline{scope}. The \lstinline{execute} function gets called by \lstinline{Authn} as it runs the domain specific program line by line.
%% \fi
%% 

The \lstinline{retrieve} function of a DSL operation is used to extract a return value from the operation. Whenever a DSL operation is passed as an argument to another operation, the parent operation calls the child's \lstinline{retrieve} in order to resolve the return value. Since these operations are not top-level, they may not affect the format string. Therefore, they do not receive the \lstinline{formatVars} array, only the \lstinline{scope}.

Take the following example from Gogs: \lstinline{SetContextVar("repo", Get("id"))}. The outer \lstinline{SetContextVar} calls \lstinline{retrieve} on the inner \lstinline{Get("id")} to resolve the its return value.

%% 
%% \iffalse
%%  of the ID within the request payload.

%% In contrast to the \lstinline{execute} function, the \lstinline{retrieve} function is called by other operations directly. 
%% \fi
%% 

The function type of \lstinline{retrieve} is produce in Code Snippet~\ref{code:retrieveFunctionType}.

\begin{lstlisting}[float=h,label=code:retrieveFunctionType,caption=The function type of the \lstinline{retrieve} function.]
retrieve(r *ExtendedRequest, scope scopeContainer) interface{}
\end{lstlisting}

When a DSL operation implements these two functions, it enables a hierarchical domain specific language. A DSL operation may invoke other operations as arguments to it or be an argument to another operation. This chaining enables flexible domain specific programs.

%% 
%% \iffalse
%% The \lstinline{retrieve} may use the HTTP request at hand and the current \lstinline{scope} to produce some return result.

%% be invoked by another operation.
%% \fi
%% 
