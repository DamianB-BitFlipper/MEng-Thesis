%% This is an example first chapter.  You should put chapter/appendix that you
%% write into a separate file, and add a line \include{yourfilename} to
%% main.tex, where `yourfilename.tex' is the name of the chapter/appendix file.
%% You can process specific files by typing their names in at the 
%% \files=
%% prompt when you run the file main.tex through LaTeX.
\chapter{Related Work}

% TODO: Maybe explain what external observation and internal observation are

The state of the art for using hardware devices for website authentication is two-factor authentication. Websites that support two-factor authentication simply request the secondary mode of authentication during login, in addition to supplying a password. Two-factor authentication solves the security problems for login quite well \cite{TODO-quest-to-remove-passwords}. Depending on the implementation of the specific hardware device, most provide strong resilience to impersonation, external observation, internal observation, leakage of data secrets, third-part trust, etc. However, as explained previously, these benefits do not pertain beyond the long point of a website and assume a weaker threat model. The adversary could simply wait for the user to faithfully log in, and then launch their planned attack. 

This research builds on top of the description of transaction authentication as outlined by the first version of the webauthn specification \cite{webauthn}. The webauthn specification is a comprehensive document, standardizing a protocol for two-factor authentication with configurable extensions. The protocol covers in detail how traditional two-factor authentication should be performed under webauthn as well as details on how to extend it to perform transaction authentication. Transaction authentication is similar to a regular two-factor authentication event, just with some additional user involvement; the user must confirm a given transaction on a dedicated hardware device for the authentication to proceed. 

The purpose of transaction authentication in the eyes of the webauthn specification authors is to provide integrity to high risk operations on a website. In fact, further papers including \cite{EuroFIDO} detail use-cases of webauthn transaction authentication such as authorizing large value monetary transactions, consenting to data being shared with a third-party and authorizing a trusted service to sign a digital contract. All of these operations are very important and requesting individual authentication for each may be warranted. 

It appears that the state of the art for known security benefits through transaction authentication ends there. There is no research detailing how webauthn transaction authentication should be integrated into a service, what are good design choices in doing so or what the engineer should keep in mind in order to avoid accidental security holes.

\iffalse

\fi
