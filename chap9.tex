%% This is an example first chapter.  You should put chapter/appendix that you
%% write into a separate file, and add a line \include{yourfilename} to
%% main.tex, where `yourfilename.tex' is the name of the chapter/appendix file.
%% You can process specific files by typing their names in at the 
%% \files=
%% prompt when you run the file main.tex through LaTeX.

\chapter{Conclusion}\label{Chap:Conclusion}

This thesis presents a webauthn firewall architecture with the central aim of reducing the burden for integrating transaction authentication into a new or existing web service. Before this research, webauthn would be integrated directly into the web service's codebase which is time consuming and difficult to maintain. The firewall approach enables an engineer to configure a ruleset for the firewall using the domain specific language and, if necessary, custom handlers. This configuration determines which HTTP requests are high-risk and which are not. The firewall processes those that are deemed high-risk and only allows them to pass through if the transaction authentication validates successfully.

Three case studies with evaluations justify the advantages of this system over integrating webauthn intrusively.

\begin{itemize}[nosep]
\item Complexity: The average firewall configuration file size among the case studies is approximately 250 lines of code. The average secured route for Gogs done intrusively is 87 lines of code. Considering all of the boilerplate code necessary to support webauthn, the firewall approach is almost 8 times more concise than the intrusive approach.

%% 
%% \iffalse
%%  It is undeniably the less complex approach to integrating transaction authentication.

%% This does not account for all of the necessary boilerplate code which the firewall handles by default. 

%%  per secured route for Gogs done intrusively, using the firewall is undeniably the less complex approach.

%% The firewall also handles all of the necessary boilerplate code resulting 
%% \fi
%% 

\item Configurability: Among the case studies, 75\% of all routes can be secured using the domain specific language in 20 lines of code or less. The average secured route requires 11 lines of code, so making adjustments to the configuration is simple and painless. 

  %% 
%% \iffalse
%% The firewall supplies the engineer with a collection of default handlers and a domain specific language to enable ease and flexibility of configuration. 
%% \fi
%% 

\item Intrusiveness: In both RESTful case studies using the webauthn firewall, no modifications are necessary to the backend. The Gogs case study with the firewall requires 169 lines of code changes. Contrasted to the almost 1300 lines of code change over 18 different files for the intrusive Gogs case study, the firewall is the significantly less invasive approach.

  %% 
%% \iffalse
%% to integrate webauthn with the firewall. Using the firewall with the server-side rendering Gogs web service requires only 169 changes, mainly for context retrieval support. Intrusively integrating webauthn in Gogs which requires almost 1300 code changes over 18 different files. In contrast to 169 lines of code changes, the firewall is significantly much less invasive to the web service backend.
%% \fi
%% 

\end{itemize}

The firewall does incur a performance penalty which varies depending on the implementation details of the web service. The session ID implementation of Gogs requires that the firewall make an HTTP request in order to identify the current user issuing a request. Under heavy concurrent load, the latency penalty of the additional HTTP request is as high as 70 milliseconds.

In conclusion, the webauthn firewall architecture is flexible and extensible. With relatively little effort, it can equip even a production level web service with webauthn transaction authentication. When well tuned for performance, such a webauthn firewall architecture design is an appealing option for greater web security.

%% 
%% \iffalse
%% for integrating transaction authentication into a new or existing web service. Using this approach dramatically decreases the code complexity for integrating webauthn 

%% This thesis presents a webauthn firewall architecture and three case studies with evaluations to justify the benefits and advantages of this system. 

%% From the viewpoint of code complexity the firewall 

%%  presents an attractive system architecture to integrating webauthn transaction authentication into a new or existing web service. 
%% \fi
%% 
