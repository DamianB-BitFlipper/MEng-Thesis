% $Log: abstract.tex,v $
% Revision 1.1  93/05/14  14:56:25  starflt
% Initial revision
% 
% Revision 1.1  90/05/04  10:41:01  lwvanels
% Initial revision
% 
%
%% The text of your abstract and nothing else (other than comments) goes here.
%% It will be single-spaced and the rest of the text that is supposed to go on
%% the abstract page will be generated by the abstractpage environment.  This
%% file should be \input (not \include 'd) from cover.tex.

\iffalse
by requiring attestation from a hardware authenticator device. 

may have complete control of all infrastructure components: the host computer, operating system, web-browser, network, etc. 
\fi


Transaction authentication is an attractive extension to typical two-factor authentication. It is proposed in the WebAuthn standard by the World-Wide-Web Consortium (W3C) as a mechanism to secure individual ``high-risk'' operations of a website via a hardware authenticator device. It defends against a strict threat model where an adversary can modify or create HTTP requests between the user and the web service.

Unfortunately, transaction authentication as defined by WebAuthn is not yet adopted in practice. Demo applications exist, but have significant modifications to their codebases in order to support transaction authentication. This approach comes with a degree of complexity, which is error-prone and difficult to develop and maintain.

%% 
%% \iffalse
%% is difficult to develop, maintain and is error-prone due to the entailed complexity.
%% \fi
%% 

This thesis presents a firewall system for integrating transaction authentication into a new or existing web service. The aim is to lower the barriers preventing its adoption. An engineer would need to populate a configuration file for the firewall and make relatively few code changes to the web service to integrate transaction authentication. The firewall intercepts all HTTP traffic sent to the web service. Based on the configuration, any requests deemed innocuous are proxied directly to the web service. All other requests are considered high-risk and are held back and validated using transaction authentication. Only if the validation passes are they also permitted to pass through to the web service.

This thesis evaluates the footprint and complexity of the firewall approach. It is close to 8 times more concise than the typical means for integrating transaction authentication in a web service. This entails easier development and fewer opportunities for accidental bugs or errors in the code. However, under heavy load, there is an associated latency of at worst 1.5x slower when using the firewall when compared to accessing the original web service directly without WebAuthn.
