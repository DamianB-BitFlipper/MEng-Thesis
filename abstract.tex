% $Log: abstract.tex,v $
% Revision 1.1  93/05/14  14:56:25  starflt
% Initial revision
% 
% Revision 1.1  90/05/04  10:41:01  lwvanels
% Initial revision
% 
%
%% The text of your abstract and nothing else (other than comments) goes here.
%% It will be single-spaced and the rest of the text that is supposed to go on
%% the abstract page will be generated by the abstractpage environment.  This
%% file should be \input (not \include 'd) from cover.tex.
Transaction authentication is an attractive extension to typical two-factor authentication. It is proposed by the World-Wide-Web Consortium (W3C) as a mechanism to secure individual ``high-risk'' operations of a website by requiring attestation from a hardware authenticator device. Unfortunately, transaction authentication is not widely adopted. Typical means for supporting transaction authentication in a web service involve significant modifications to its codebase.

This thesis presents a firewall system for integrating transaction authentication into a new or existing web service. The aim is to lower the barriers preventing its adoption. An engineer would need to populate a configuration file for the firewall and make relatively few code changes to the web service to integrate transaction authentication. 

The firewall monitors HTTP traffic sent to the web service. Base on the configuration, any requests deemed to be of high-risk are held back and validated using transaction authentication. If the validation passes, then the requests are permitted to pass through to the web service.

This thesis evaluates the footprint and complexity of the firewall approach. It is close to 8 times more concise than the typical means for integrating transaction authentication in a web service. This entails easier development and fewer opportunities for accidental bugs or errors in the code.
